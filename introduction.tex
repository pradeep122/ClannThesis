\chapter{Introduction}
%\label{ch:intro}
\section{Motivation}
The Semantic web\footnote{\url{http://www.w3.org/2001/sw/}} aims to simplify the process of building knowledge-based applications by enabling a web of inter-operable and machine-readable data.  This is done by formalizing the descriptions of the structure and semantics of the data available on the web.  The linked data initiative\footnote{\url{http://linkeddata.org/}} is a positive step in that direction, exposing huge amounts of data for further analysis and use by other applications.  However creating and exposing linked data is a task that requires thorough knowledge of various technologies which would be a huge hassle for the novice users.  A solution to this is to create technologies which would enable the average internet user to annotate and embed data in his/her own textual resources.  This work aims to explore the possibility of using controlled natural languages(hereby referred to as CNL) as an interface for semantic annotation, specifically targeting the domain of project documents like meeting minutes,  status reports, etc.  The major goal was to enable novice users to author and annotate text documents simultaneously using a controlled language.  Furthermore, these documents can be parsed to extract the implicit knowledge contained, due to the enforcement of a fixed grammar and vocabulary. This completes the task of converting human-readable controlled language texts to machine-interpretable structured information which could be further exploited. Previously this approach was used to build an annotation tool along with prototypes of the grammar and ontologies for the meeting minutes domain \cite{cnl09}.

\section{Research Goal}
Using CNLs as interfaces to knowledge acquisition applications, soothens the barriers of entry for novice users thereby leading to a greater public involvement.  

The main contributions of the thesis include 
\begin{itemize}

\item{An Annotation Software platform for the knowledge acquisition and management pertaining to the meeting minutes domain}
\item{The development and publishing of an  ontology which models the domain of project documents}
\item{Development of the CLANN grammar using link grammars and the corresponding parser}
\item{A GWT based web interface for the novice users to aid them in authoring texts in Controlled language}
\end{itemize}  

\section{Thesis Layout}

